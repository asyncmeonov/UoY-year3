%--------------------
% Packages
% -------------------
\documentclass[11pt,a4paper]{article}
\usepackage[utf8]{inputenc}
\usepackage[T1]{fontenc}
\usepackage{outlines}
%\usepackage{gentium}
\usepackage{mathptmx} % Use Times Font


\usepackage[pdftex]{graphicx} % Required for including pictures
\usepackage[pdftex,linkcolor=black,pdfborder={0 0 0}]{hyperref} % Format links for pdf
\usepackage{calc} % To reset the counter in the document after title page
\usepackage{enumitem} % Includes lists

\frenchspacing % No double spacing between sentences
\linespread{1.2} % Set linespace
\usepackage[a4paper, lmargin=0.1666\paperwidth, rmargin=0.1666\paperwidth, tmargin=0.1111\paperheight, bmargin=0.1111\paperheight]{geometry} %margins
%\usepackage{parskip}

\usepackage[all]{nowidow} % Tries to remove widows
\usepackage[protrusion=true,expansion=true]{microtype} % Improves typography, load after fontpackage is selected
\usepackage{csquotes}
\usepackage[style=verbose-ibid,backend=bibtex]{biblatex}
\bibliography{bibliography}

%-----------------------
% Set pdf information and add title, fill in the fields
%-----------------------
\hypersetup{ 	
pdfsubject = {},
pdftitle = {INNS Open Assessment},
pdfauthor = {}
}

\title{INNS Open Assessment}

\author{Y3843100}

\date{\today}
%-----------------------
% Begin document
%-----------------------
\begin{document}
% ! ============
% ! Terminology
% ! ============
% CTG - Cardiotocogram
% FHR - fetal heart rate
% STM - short term variability::  beat-to-beat differences between consecutive heart beats
% LTM - long term variability:: variations in the interval length over N R-R intervals
% R-R interval - beat-to-beat time.


\maketitle


\section{[20 marks] Discussion of architectures.}

This section should:
\begin{outline}
  \1 describe (briefly) the data you have, and how much there is of it.
  \1 identify the type of problem
  \1 identify which classes of architectures would be suitable
  \1 give a brief discussion of the technical features of the architectures, and the advantages and disadvantages of each
  \1 state which class of architecture you are going to use and justify your choice, relating the characteristics of the problem to the advantages/disadvantages of the architecture.
\end{outline}

To do this you might need to:
\begin{outline}
  \1 do some preliminary experiments with simple versions of the architecture to get a feel for what will work
  \1 do some exploratory data analysis to see what the characteristics of the data are
  \1 consider the principles involved and relate them to the problem.
\end{outline}

\paragraph{}
The dataset contains 2126 fetal cardiotocograms (CTGs) from different patients each of which has 23 different recorded features. The CTGs have been annotated by three expert obstetricians creating two categories of classess \autocite{Campos:2000}. One is a 10 tuple with respect to the fetal heart rate FHR patterns and the other is a three tuple regarding fetal state. This gives us two classification problems with respectivelly 10 and 3 distinct classes.

%architectures to consider
% single layer perceptron :: 
%  + simple and interpretable
%  - does not handle high dimentional data. Works only if our data is linearly separable.
% multi-layer perceptron

For both the 10 and 3 class problems we can rule out the single layer perseptron as its bipartitioning ability can discriminate up to two classes.
With 23 complex features, we can immediately rule out single perceptron models as they do not handle high dimensional data.  



\section{[40 marks] Creation and application of neural networks.}
This section should
\begin{outline}
    

  \1 Describe the chosen inputs to (and outputs from) the networks.
  \1 Describe how the data you started with have been preprocessed.
  \1 Give sufficient detail for someone else to process a new batch of data for use with the final trained network.
  \1 State which training algorithm you selected, and explain how you selected that training algorithm. For this training algorithm, give sufficient detail to enable someone to use the same training algorithm in exactly the same way. This does NOT mean (for example) describing gradient descent in great detail. It DOES mean giving any parameters, initialisation, etc, even if they are the toolbox defaults.
  \1 Explain the process you went through in making the selection of the final architecture, for example, the number of neurons or the number of layers to use.
\end{outline}


To do this you might need to:
\begin{outline}
  \1 Test of one or more networks to demonstrate the effect of different
preprocessing choices on the performance of the network.
  \1 Try different training algorithms on one or more networks to compare
performance.
  \1 Evaluate a number of networks, and record details of their structures and how
\end{outline}

\section{[20 marks] Results and evaluation}
This section should
\begin{outline}
  \1 Explain the metric or metrics you have used for comparison between networks.
  \1 Give a synopsis of the results obtained from the final selected network.
  \1 Evaluate the results, in relation to the problem posed in the scenario.
To do this you might need to:
  \1 Consider different metrics for performance, appropriate to the problem.
Remember that a Mean Squared Error (MSE) on its own is not always helpful in
judging how well something works.
  \1 Identify anything of interest in the results, such as areas of particularly good or poor performance, or variation between different training runs.
  \1 Reflect on the conclusions that you may draw from the results, and whether
they are showing that the neural network is useful in this case.
\end{outline}
\section{[20 marks] Further application}
In the previous sections you used a neural network to convert cardiotocogram features
into a diagnosis. Another tool for detection and diagnosis of fetal abnormalities is the ultrasound scan, that produces an image of the a section through the fetus. Interpreting fetal scans is a highly complex task which require years of training. Assume the availability of a large number of fetal scan images, both normal and with some abnormality, and labelled to indicate different types of abnormalities. The task is for a neural network to process new ultrasound images, and to indicate which images needed further investigation. This section should discuss the issues you would need to consider in relation to:
\begin{outline}
  \1 selection of an architecture
  \1 construction of the network
  \1 use of data for training
  \1 evaluation of the network
\end{outline}
\end{document}
